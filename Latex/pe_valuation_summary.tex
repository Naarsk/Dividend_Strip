
\documentclass[12pt]{article}
\usepackage{amsmath, amssymb}
\usepackage{geometry}
\usepackage{amsfonts}
\geometry{margin=1in}

\title{Summary and Methodology of\\\textit{Valuing Private Equity Investments Strip by Strip}}
\author{}
\date{}

\begin{document}
\maketitle

\section{Overview}
This document summarizes the paper \textit{Valuing Private Equity Investments Strip by Strip} by Arpit Gupta and Stijn Van Nieuwerburgh. The paper introduces a new method to value private equity (PE) investments by constructing replicating portfolios of publicly traded securities, namely zero-coupon bonds and equity strips. These replicating portfolios match the cash flow timing and risk characteristics of private equity funds.

\section{Replicating Strips: Construction of Factors}
The paper constructs a set of public market instruments, or ``strips'', which represent the priced risk factors used to value PE. These include:

\begin{itemize}
    \item Zero-coupon bonds
    \item Dividend strips on aggregate and factor-sorted portfolios
    \item Capital gain strips on aggregate and factor-sorted portfolios
\end{itemize}

Let $F_{t,t+h}$ denote the $K \times 1$ vector of cash-flow realizations over horizon $h$, where $K$ is the number of risk factors. Each element of $F_{t,t+h}$ corresponds to a specific strip:

\begin{equation}
F_{t,t+h} = \begin{bmatrix}
1 \\
\frac{D^m_{t+h}}{D^m_t} \\
\frac{P^m_{t+h}}{P^m_t} \\
\vdots \\
\frac{D^f_{t+h}}{D^f_t} \\
\frac{P^f_{t+h}}{P^f_t}
\end{bmatrix}
\end{equation}

Where:
\begin{itemize}
    \item $D^m_{t}$ is the dividend on the market portfolio at time $t$
    \item $P^m_{t}$ is the price of the market portfolio at time $t$
    \item $D^f_{t}$ and $P^f_{t}$ refer to the dividend and price of factor portfolios (e.g., value, momentum)
\end{itemize}

\section{Linking PE Cash Flows to Strips}
PE fund cash flows are decomposed into operating cash flows and exit cash flows. These are mapped to dividend and gain strips respectively:

\begin{itemize}
    \item \textbf{Operating income} from portfolio firms maps to \textit{dividend strips}
    \item \textbf{Exit proceeds} from asset sales map to \textit{capital gain strips}
\end{itemize}

\section{Linear Pricing Model}
Let $C_{i,t+h}$ be the cash flow of PE fund $i$ at time $t+h$. The cash flow is priced as a linear function of the strip payoffs:

\begin{equation}
\mathbb{E}[C_{i,t+h}] = \beta_i^\top F_{t,t+h}
\end{equation}

The model estimates $\beta_i$, the fund’s exposure to each strip.

\section{Total Set of Strips}
The full model includes:

\begin{itemize}
    \item 1 zero-coupon bond
    \item 7 dividend strips (market + 6 factors)
    \item 7 gain strips (market + 6 factors)
\end{itemize}

Total of $K = 15$ risk factors, each with $H = 64$ horizons (quarters), yielding 960 distinct priced strips.

\section{Estimation Approach}
The authors estimate the pricing kernel using GMM, exploiting both the time series and cross-sectional variation in PE fund returns.

\subsection{Valuation}
Valuation of a PE fund is achieved by summing the present value of expected future cash flows weighted by the strip prices:

\begin{equation}
\text{Value}_{i,t} = \sum_{h=1}^{H} \beta_i^\top F_{t,t+h} \cdot \text{Price}(F_{t,t+h})
\end{equation}

\section{Conclusion}
This methodology allows for a transparent, replicable approach to valuing PE investments, linking them directly to publicly observable risk factors. It provides insights into the sources of PE returns and their risk exposures over time.
\end{document}
